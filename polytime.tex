\documentclass[manuscript]{acmart}
\begin{document}

    \title{A Polynomial Time Algorithm for 3SAT} % Sets article title
    \author{Robert Quigley}
    % what should I do for \affiliation?
    \email{robertquigley21@gmail.com}
    % do not use \thanks for journals
    % see the acks environment sec 2.13
    % remember to put the CCS codes in!!
    
    \begin{abstract}
        This paper includes the presentation of a polynomial time algorithm to
        solve 3SAT. It begins with the problem definition and format of various
        parts of the problem. It then offers various lemmas of important aspects of 
        the 3SAT problem along with their corresponding proofs. Next, it describes
        the algorithm that uses these lemmas to solve 3SAT. And finally, it proves
        that the algorithm works for any general case of 3SAT. The algorithm relies
        on the fact that an instance of 3SAT is unsatisfiable iff contradicting
        1-terminal clauses can be derived in polynomial time.
    \end{abstract}
    
    % TODO: google a list of keywords for ACM journals
    \keywords{test1, test2}
    
    \maketitle % creates title using information in preamble (title, author, date)
    
    % \acmplain for theorem, conjecture, proposition, lemma, corollary, etc
    % \acmdefinition for example and definition
    % \acks for acknowledgements
    % bibtex or biblatex for bibliography
    
    TODO: find who the heck wrote '3SAT is NP-complete'
    read https://dl.acm.org/doi/10.1145/800157.805047
    
    \section{Definitions}

    \begin{definition}
        Terminals: symbols used in the 3SAT problem that can be assigned a value
        of either 0 or 1, True or False, or any other binary assignment. They usually
        take the form $x_i$ where $i$ is a natural number.
    \end{definition}
    \begin{definition}
        Clauses: a set of terminals combined by logical or operators. Each 
        terminal can also be negated. Clauses usually take the form 
        $(x_i \lor \neg x_j \lor x_k)$ where $x_i \neq x_j \neq x_k$
    \end{definition}
    \begin{definition}
        (3SAT) Instance: a set of any number of clauses combined by logical
        and operators. Note that entire clauses may not be negated. Terminals may
        not repeat within a clause, but they are free to repeat between clauses. 
        Instances usually take the form:
        $(x_i \lor \neg x_j \lor x_k) \land (x_l \lor x_m \lor x_n)$
    \end{definition}
    \begin{definition}
        Assignment: A list of values in which each value represents either True
        or False such that each item in the list corresponds to a terminal and 
        all terminals are assigned a value.
    \end{definition}
    \begin{definition}
        Satisfying Assignment: An assignment, $A$, is said to satisfy the instance, 
        if applying $A$ will make the instance evaluate to True.
    \end{definition}
    \begin{definition}
        The 3SAT Problem: Given a 3SAT instance, does there exist a satisfying assignment?
    \end{definition}
    \begin{definition}
        Blocking an Assignment: An assignment, $A$, is said to be blocked if, given a
        clause, $C$, there is no way that $A$ can satisfy the instance.
    \end{definition}
    \begin{definition}
        TODO: define a more 'specifically in polytime' type of implication
        Implication: A clause, $C$, is said to imply another clause, $D$, if all 
        assignments blocked by $D$ are also blocked by $C$ 
    \end{definition}
    \begin{definition}
        Given Clauses: clauses which were given in the original instance.
    \end{definition}
    \begin{definition}
        Implied Clauses: clauses which are implied by the clauses in the original instance.
    \end{definition}
    \begin{definition}
        k-terminal (k-t) clause: a clause is described as a k-terminal or a k-t clause
        if there are $k$ terminals in the clause
    \end{definition}

    \section{Reformatting}

    Since there are a lot of constant characteristics about an instance of 3SAT, 
    we can remove most of them to allow ourselves to focus only on what changes
    from instance to instance. A list of unchanging characteristics follows:
    \begin{itemize}
        \item the symbol $x$
        \item logical and operators
        \item logical or operators
    \end{itemize}
    
    The only difference between instances, therefore, is the subscript of the terminal.
    The following items will be changed to improve compatibility with python code:
    *possible footnote to describe how to represent it as list of lists
    \begin{itemize}
        \item parentheses will become square brackets
        \item negation symbols will become minus signs
        \item an instance may be surrounded with square brackets to show it is a list of lists
    \end{itemize}
   
    For example, the instance:
    
    $(\neg x_a \lor x_b \lor x_c) \land (x_a \lor x_d \lor x_e)$

    will be written as:

    $[[-a, b, c], [a, d, e]]$


    \section{Lemmas}

    \subsection{Lemma A}
    \subsection{Lemma B}
    \subsection{Lemma C}
    \subsection{Lemma D}
    \subsection{Lemma E}

    \section{Algorithm}

    \section{Time Complexity Analysis}

    \section{Proof of Correctness}

    \section{Conclusion}

    TODO:
    \begin{itemize}
        \item reorder definitions 
    \end{itemize}


\end{document} % This is the end of the document
