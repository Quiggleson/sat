\documentclass[manuscript]{acmart}

\usepackage{graphicx}

\graphicspath{{./Source/}}

\begin{document}

    \title{A Polynomial Time Algorithm for 3SAT} 
    \author{Robert Quigley}
    \email{robertquigley21@gmail.com}

    \begin{abstract}
        It is shown that any two clauses in an instance of 3SAT sharing the same terminal which is positive in one clause and negated in the other can imply a new clause composed of the remaining terms from both clauses. Clauses can also imply other clauses as long as all the terms in the implying clauses exist in the implied clause. It is shown an instance of 3SAT is unsatisfiable if and only if it can derive contradicting 1-terminal clauses in exponential time. It is further shown that these contradicting clauses can be implied with the aforementioned techniques without processing clauses of length 4 or greater, reducing the computation to polynomial time. Therefore there is a polynomial time algorithm that will produce contradicting 1-terminal clauses if and only if the instance of 3SAT is unsatifiable. Since such an algorithm exists and 3SAT is NP-Complete, we can conclude P = NP.
    \end{abstract}
    
    \keywords{NP-Complete, SAT, 3SAT, Satisfiability, P vs NP, Polynomial Time, Non-deterministic Polynomial Time, Complexity Theory, Computational Complexity, Complexity Classes, Cook-Levin Theorem, Karp's 21 NP-Complete Problems}
    
\begin{CCSXML}
        <ccs2012>
           <concept>
               <concept_id>10003752.10003777.10003778</concept_id>
               <concept_desc>Theory of computation~Complexity classes</concept_desc>
               <concept_significance>500</concept_significance>
               </concept>
        </ccs2012>
\end{CCSXML}
        
    \ccsdesc[500]{Theory of computation~Complexity classes}
    
    \maketitle 
    
    \section{Introduction}

        This section introduces the 3SAT problem, the implications of
        solving it in polynomial time, and the structure of the paper.

        As seen in \cite{10.1145/800157.805047},
        The boolean satisfiability problem is given as a set of terminals, $x_1, x_2, ..., x_n$,
        each of which can be assigned a value of True or False,
        combined by logical AND operators, logical OR operators, and negations.
        The problem is to determine whether or not there exists an assignment
        for each terminal that allows the instance to evaluate to True.
        As seen in \cite{Karp1972}, the satisfiability problem
        with at most 3 literals per clause is NP-Complete. This is the same 
        as the boolean satisfiability problem, but it is presented such that
        a clause contains exactly three terminals combined with logical OR operators
        and possbily negations, and each clause is combined with logical AND operators.
        The idea of NP-completeness shows that if one NP-complete problem can be solved
        in polynomial time, then all problems in the class NP can be solved in polynomial time.
        In other words, P = NP. 

        The paper is structured as follows:
        \begin{enumerate}
            \item Introduction
            \item Standard Definitions
            \begin{itemize}
                \item Terms relating to the 3SAT problem
            \end{itemize}
            \item Algorithm-Specific Definitions
            \begin{itemize}
                \item Terms relating to specific aspects of 3SAT regularly referenced in this paper
            \end{itemize}
            \item Reformatting and Processing
            \begin{itemize}
                \item Defines how clauses and instances of the 3SAT problem will appear within the paper
            \end{itemize}
            \item Lemmas
            \begin{itemize}
                \item A list of lemmas and their proofs pertaining to the algorithm
            \end{itemize}
            \item Algorithm
            \begin{itemize}
                \item A step-by-step description of the algorithm to solve 3SAT in polynomial time
            \end{itemize}
            \item Time Complexity Analysis
            \begin{itemize}
                \item A step-by-step analysis of the time complexity of the algorithm
            \end{itemize}
            \item Proof of Correctness
            \begin{itemize}
                \item A proof showing the algorithm works for every instance of 3SAT
            \end{itemize}
            \item Conclusion
            \item References
        \end{enumerate}
        
    \section{Standard Definitions}

    \begin{definition}
        Terminals: symbols used in the 3SAT problem that can be assigned a value
        of either 0 or 1, True or False, or any other binary assignment. They usually
        take the form $x_i$ where $i$ is a natural number.
    \end{definition}
    \begin{definition}
        Terms: terminals in either the positive or negated form that appear in a clause.
        If a term is positive, then the value of the terminal will be the 
        same as the value of the term. If a term is negated, then the value
        of the terminal will be the opposite of the value of the term.
    \end{definition}
    \begin{definition}
        Clauses: a set of terms combined by logical OR operators.
        Clauses usually take the form 
        $(x_i \lor \neg x_j \lor x_k)$ where $x_i \neq x_j \neq x_k$
    \end{definition}
    \begin{definition}
        (3SAT) Instance: a set of any number of clauses combined by logical
        AND operators. Terminals may not repeat within a clause
        \footnote{See Lemma 5.3}
        , but they are free to repeat between clauses. 
        Instances usually take the form:

        $(x_i \lor \neg x_j \lor x_k) \land (x_l \lor x_m \lor x_n)$
    \end{definition}
    \begin{definition}
        Assignment: A list of values in which each value represents either True
        or False such that each item in the list corresponds to a terminal and 
        all terminals are assigned a value.
    \end{definition}
    \begin{definition}
        Satisfying Assignment: An assignment, $A$, is said to satisfy the instance, 
        if applying $A$ will make the instance evaluate to True.
    \end{definition}
    \begin{definition}
        Satisfiable: an instance is satisfiable iff there exists a satisfying assignment.
    \end{definition}
    \begin{definition}
        Unsatisfiable: an instance is unsatisfiable iff there does not exist a satisfying assignment.
    \end{definition}

    \section{Algorithm-Specific Definitions}
    \begin{definition}
        Blocking an Assignment: An assignment, $A$, is said to be blocked by a clause, $C$ if, 
        given an instance containg $C$, there is no way that $A$ allows $C$ to evaluate to True, and
        thus there is no way $A$ allows the instance to evaluate to True.
    \end{definition}
    \begin{definition}
        Implication: A clause, $C$, is said to imply another clause, $D$, if all 
        assignments blocked by $D$ are also blocked by $C$.
    \end{definition}
    \begin{definition}
        Given Clauses: clauses which were given in the original instance.
    \end{definition}
    \begin{definition}
        Implied Clauses: clauses which are implied by the clauses in the original instance.
    \end{definition}
    \begin{definition}
        k-terminal (k-t) clause: a clause is described as a k-terminal or a k-t clause
        if there are $k$ terminals in the clause.
    \end{definition}
    \begin{definition}
        Reduction: A special type of implication in which the implied 
        clause is shorter than the implying clause(s).
    \end{definition}
    \begin{definition}
        Expansion: A special type of implication in which the 
        implied clause is longer than the implying clause(s) and all terms in the implying
        clause exist in the implied clause.
    \end{definition}
    \begin{definition}
        Contradicting Clauses: A set of clauses is said to be contradicting if
        they block every assignment.
    \end{definition}

    \section{Reformatting and Processing}

    Since there are a lot of constant characteristics about an instance of 3SAT, 
    we can remove most of them to allow ourselves to focus only on what changes
    from instance to instance. A list of unchanging characteristics follows:
    \begin{itemize}
        \item the symbol $x$
        \item logical AND operators
        \item logical OR operators
    \end{itemize}
    
    The only difference between instances, therefore, is the subscript of the terminal.
    
    Additionally, the following items will be changed to improve compatibility
    with the Python programming language wherein an instance is expressed as
    a list of lists and each inner list represents a clause:
    \begin{itemize}
        \item parentheses will become square brackets
        \item negation symbols will become minus signs
        \item an instance may be surrounded with square brackets to show it is a list of lists
    \end{itemize}
   
    For example, the instance:
    
    $(\neg x_a \lor x_b \lor x_c) \land (x_a \lor x_d \lor x_e)$

    will be written as:

    $[[-a, b, c], [a, d, e]]$

    Instances of 3SAT will further be processed by removing any clauses that do not block
    any assignments. Since this algorithm relies on implications of new clauses, if we ever
    come across a clause that blocks no assignments, then by definition it will not be able
    to imply any additional clauses that block at least one assignment.

    As such, we will ignore any clauses given or derived that are described in Lemma 5.3

    \section{Lemmas}

    \begin{lemma}
        A clause can block an assignment    
    \end{lemma}
    \begin{proof}
        Recall the definition of a clause blocking an assignment:
        An assignment is blocked if it does not allow the clause to evaluate to True.

        Since terms in a clause are combined by logical OR operators, a 
        clause cannot evalute to True iff all terms evalute to False.

        A term evaluates to False if it's either (1) negated and the terminal's
        value is True or (2) positive and the terminal's value is False

        Given a clause, we know there are some number of unique terminals and we can
        find assignments where all terms evalute to False.

        Since all the terms evaluate to False and are combined by logical OR operators, 
        the clause will evalutae to False.

        Since the instance is composed of clauses combined by logical AND operators 
        and one clause evalutes to False, then the entire instance evalutes to False.

        Since the instance evaluates to False, the assignment cannot satisfy
        the instance.
    \end{proof}

    \begin{lemma}        
        For a given instance with $n$ terminals, there are $2^n$ possible assignments.
    \end{lemma}
    \begin{proof}
        An assignment for this instance consists of $n$ values, each with two possible
        values, True or False.

        Therefore, there are $2^n$ possible assignments.
    \end{proof}

    \begin{lemma}
        If a clause contains the same terminal in its negated and positive form, it will
        not block any assignments.
    \end{lemma}
    \begin{proof}

        Consider a clause containing the same terminal in both the positive and
        negative form.

        We know that terminal must either be True or False. Consider both cases:

        That terminal's value is True:
        The positive form of the terminal will be True and the clause will evaluate to True.

        That terminal's value is False:
        The negated form of the terminal will be True and the clause will evaluate to True.

        Since the clause evaluates to True in every case, there will never be an 
        assignment with which it is impossible to make this clause evalute to False.
    \end{proof}

    \begin{lemma}
        Each clause of length $k$ blocks $2^{n-k}$ assignments.
    \end{lemma}
    \begin{proof}
        Consider the generic $k$-terminal clause, $C$, in an instance with
        $n$ terminals.

        As seen in Lemma 5.1, this blocks all assignments where all of the terms
        evaluate to False.

        Since the values for $k$ terminals are set, there are $n-k$ terminals left
        whose values could be True or False.

        Since an assignment exists for every possible way to assign values to
        each terminal, we know an assignment exists for every possible way to
        assign a value for these $n-k$ terminals.

        There are two possible ways to assign values to each of these $n-k$ terminals
        so there are $2^{n-k}$ unique assignments blocked by $C$.
    \end{proof}

    \begin{lemma}
        For any clause, $C$, if we select a terminal, $t$, that's not in $C$ 
        then half of the assignments
        blocked by $C$ will assign True to $t$ and the other half will assign 
        False to $t$.
    \end{lemma}
    \begin{proof}
        We have a clause, $C$, of fixed yet arbitrary length:

        $[a, b, c, ...]$

        Now we select a terminal, $t$, that's not in $C$, say we call this
        terminal $t$.

        Want to show half of the assignments blocked by $C$ assign True to 
        $t$ and the other half assign False to $t$.

        We know there will be no overlap between these assignments because a single assignment
        cannot assign both the values True and False to the same terminal.

        Now we just have to show that exactly half of the assignments are blocked 
        by assigning either True or False to $t$.

        Let's say there are $k$ terms in C, then we know it blocks assignments 
        where all $k$ terms evaluate to False.

        By Lemma 5.4, this clause blocks $2^{n-k}$ assignments. 

        If we fix the value of $t$, then there are only $n-k-1$ terminals
        whose values could be 0 or 1. Since we have two choices per terminal
        and there are $n-k-1$ terminals, then there are $2^{n-k-1}$ assignments
        blocked by $C$ where the value of $t$ is fixed.

        Divide to get the ratio of the number of assignments blocked by adding 
        $t$ to the number of assignments blocked by $C$ without $t$:
        
        $2^{n-k-1}/2^{n-k}$

        = $2^{n - k - 1 - (n - k)}$

        = $2^{-1}$

        = $1/2$

        This shows that half of the assignments blocked by $C$ assign a fixed
        value to $t$.

        Since there are two possible values for $t$ and each block mutually
        exclusive halves of the assignments blocked by $C$, the lemma holds.

        For any clause, $C$, if we select a terminal, $t$, that's not
        in $C$ then half of the assignments blocked by $C$ will assign True
        to $t$ and the other half will assign False to $C$.
    \end{proof}

    \begin{lemma}
        Given a clause, $C$, and another clause, $D$, such that all of the 
        terms in $C$ also exist in $D$, then all of the assignments
        blocked by $D$ are also blocked by $C$.
    \end{lemma}
    \begin{proof}
        Given a clause, $C$, of a fixed yet arbitrary length:

        $[a, b, c, ...]$

        And another clause, $D$, containing all the terms in $C$ with possibly
        additional terms:

        $[a, b, c, ..., d, e, f, ...]$

        Want to show all the assignments blocked by $D$ are also blocked by $C$.

        We know that $C$ blocks all assignments that cause all the terms to evaluate
        to False.

        In other words, $C$ blocks all assignments where 
 
        $a = b = c = ... = False$

        Similarly, $D$ blocks all assignments where

        $a = b = c = ... = d = e = f = ... = False$

        Clearly all assignments consistent with the terminal assignments from $D$
        are also consistent with the terminal assignments from $C$.

        Therefore, every assignment blocked by $D$ is also blocked by $C$.
    \end{proof}

    \begin{lemma}[Reduction]
        Given the following conditions:
        \begin{itemize}
            \item $A$ is a clause of length $k$
            \item $B$ is a clause of length $k$
            \item $A$ and $B$ share $k-1$ identical terms
            \item $A$ and $B$ share one terminal that is negated in one clause
            and positive in the other
            \item $C$ is a clause of length $k-1$ composed of the shared terms
            from $A$ and $B$
        \end{itemize}
        Then $A$ and $B$ imply $C$.
    \end{lemma}
    \begin{proof}
        Given clauses consistent with the description:

        $A := [a, b, c, ... i]$

        $B := [a, b, c, ..., -i]$

        where $a, b, c, ...$ is shared between the clauses and $i$ is a 
        terminal not in $a, b, c, ...$

        Want to show this implies the described clause.

        Recall by Lemma 5.3 that the same terminal cannot appear both negated and
        positive within the same
        clause and still block an assignment, so $i$ cannot exist in $a, b, c...$

        Consider the clause

        $C := [a, b, c, ...]$

        We know by Lemma 5.5 that if we select a terminal that's not in $C$, say $t$, 
        then half of the assignments blocked by $C$ assign True to $t$ and the other
        half of the assignments blocked by $C$ assign False to $t$.

        Let this terminal $t$ that's not in $C$ be the terminal $i$ that's in $A$ and $B$.

        We know that $A$ blocks all assignments blocked by $C$ where $i$ is assigned
        the value of False.

        We know that $B$ blocks all assignments blocked by $C$ where $i$ is assigned
        the value of True.

        Since $A$ and $B$ both block mutually exclusive halves of the assignments
        blocked by $C$, we can say that $A$ and $B$ imply $C$.
    \end{proof}

    \begin{lemma}[Expansion]
        Given a clause, $C$, and a terminal, $t$, that's not in $C$, two 
        new clauses can be implied consisting of all the terms of $C$ 
        appended to either the positive form of $t$ or the negated form of $t$.
    \end{lemma}
    \begin{proof}
        Given a clause, $C$, and a terminal, $t$, that's not in $C$:

        $C := [a, b, c, ...]$

        We can compose two new clauses:

        $D := [a, b, c, ..., t]$

        $E := [a, b, c, ..., -t]$

        Since D and E both contain all the terminals from C, by Lemma 5.6 then
        D and E are implied by C.   
    \end{proof}

    \begin{lemma}[General Lemma 5.7]
        If two clauses share the same terminal, $t$, such that $t$ is 
        positive in one clause and negated in the other, then these clauses 
        imply a new clause which is composed of all the terms in both clauses 
        that do not have $t$.
    \end{lemma}
    \begin{proof}
        Consider two clauses, 

        $C := [a, b, c, ..., t]$
        
        $D := [d, e, f, ..., -t]$

        Where within a clause, the same terminal does not repeat, but between
        clauses the same terminal may repeat.

        Want to show we can imply a clause consistent with the lemma description:

        $E := [a, b, c, ..., d, e, f, ...]$

        Let's define some additional clauses:

        $E' := [a, b, c, ..., d, e, f, ..., t]$

        $E'' := [a, b, c, ..., d, e, f, ..., -t]$

        By Lemma 5.6, we know that $C$ implies $E'$, ie, all assignments blocked by
        $E'$ are also blocked by $C$.

        By Lemma 5.6, we know that $D$ implies $E''$.

        By Lemma 5.7, since $E'$ and $E''$ share all the same terms except 
        for $t$, which is positive in one clause and negated in the other, 
        we can create a new clause composed of all the shared terms in $E'$ 
        and $E''$.

        Such a clause is already defined as $E$.

        Now there are a couple extra cases to consider:
        
        \begin{itemize}
            \item There is some overlap between a, b, c, ... and d, e, f, ...
            \item There is the same terminal that's positive in a, b, c, ...
            and negated in d, e, f, ...
        \end{itemize}

        First, if the same term exists in a, b, c, ... and d, e, f, ..., then we
        can just remove one of the duplicates since one term being true implies an identical term being True.

        Secondly, if the same terminal exists, but is of the opposite form in a, b, c, ... and d, e, f, ...
        then by Lemma 5.3, this clause will always be True and thus blocks no assignments. In this case, 
        the lemma is vacuously true, but we disregard the clause as it is of no value.
        
    \end{proof}

    \begin{lemma}
        Given two clauses of lengths $k$ and $m$ that share a terminal, $t$, which is 
        positive in one clause and negated in the other, you will be able
        to directly imply clauses of length $max(k, m)-1$ to $(k + m - 2)$ where
        the function $max(a, b)$ represents the parameter with the greatest value.
    \end{lemma}
    \begin{proof}
        The smallest clause that can be implied by clauses of length $k$ and $m$
        using Lemma 5.9 occur when all but one of the terms in one clause exist in the other.

        As such, the unique terms will come from the clause that's longer.

        Removing $t$, you are left with 1 less than the maximum of $k$ and $m$.

        The largest clause can be implied if there are no terms shared between
        the two clauses. In this case you subtract $1$ from the length of each
        clause to account for $t$ and since no duplicates will be removed, the 
        resulting clause's length is $2$ less than the sum of the lengths of 
        the clauses.
    \end{proof}

    \begin{lemma}
        Given four clauses of length $k - 1,$ $A, B, C,$ and $D$, and one clause of 
        length $k,$ $E$, such that 
        \begin{itemize}
            \item $A$ and $B$ imply $E$ by Lemma 5.9
            \item $C$ and $E$ imply $D$ by Lemma 5.9
        \end{itemize}
        Then $A$, $B$, and $C$, imply $D$ by processing only clauses with a maximum length of $k - 1$.
    \end{lemma}
    \begin{proof}
        
        Consider the following figure:
        
        \begin{figure}[h]
            \includegraphics[scale=0.8]{511a}        
            \caption{A graph illustrating the derivations described by the lemma}
            \label{fig:one}
            \Description[A graph illustrating the derivations described by the lemma]
            {A graph illustrating the derivations described by the lemma}
        \end{figure}

        Define the clauses in the following manner:

        A := [a, b, $\beta$, i]

        B := [c, d, $\delta$ -i]

        C := [-a, e, f, $\phi$]

        Then the following are derived by Lemma 5.9:
        
        $E = [a, b, \beta, c, d, \delta]$ (By $A$ and $B$)

        $D = [b, \beta, c, d, \delta, e, f, \phi]$ (By $C$ and $D$)

        $F = [b, \beta, i, e, f, \phi]$ (By $A$ and $C$)

        Where $\beta, \delta, $ and $\phi$ are generic sets of terms in the instance.

        Note that since $C$ and $E$ imply $D$, then $C$ and $E$ must share the
        same terminal such that it is positive in one clause and negated in the other.
        Since all of the terms in $E$ came from $A$ or $B$ (not including $i$),
        then such a term in $C$ must have the same term of the opposite form in
        $A$ or $B$ (again it cannot be $i$ since $i$ is not in $E$). And since
        $A$ and $B$ are fixed yet arbitrary sets, it does not matter which clause
        we pick as long as it is fixed for the rest of the proof. Let's pick a
        the terminal, $a$, from clause $A$. Now $C$ contains $-a$.
 
        Recall from Lemma 5.3 that if a clause blocks any assignments, it cannot
        contain the same term in both forms, so if $\beta, \delta$, or $\phi$ contain
        the same terminal in both forms then $D$ blocks no assignments and the resulting
        clause may be disregarded.

        We know $C$ is of length $k$ and $D$ is of length $k - 1$ or $k$.

        In the following equations, let the presence of a term represent a count of one,
        and the presence of a set of terms represent the number of terms in that set. If
        multiple sets of terms are shown in parentheses, let this represent the number
        of terms found in both sets.

        Consider the case where $D$ has length $k$ then we can define $k$ in terms of $D$:
        
        $k = b + c + d + e + f + \beta + \delta + \phi - (\beta \delta) - 
        (\beta \phi) - (\delta \phi) + (\beta \delta \phi)$

        length of $F = b + i + e + f + \beta + \phi - (\beta \phi)$

        Want to show length of $F$ is less than $k$

        $b + i + e + f + \beta + \phi - (\beta \phi) < 
        b + c + d + e + f + \beta + \delta + \phi - (\beta \delta) - 
        (\beta \phi) - (\delta \phi) + (\beta \delta \phi)$

        $\rightarrow$ $i + \beta + \phi - (\beta \phi) < 
        c + d + \beta + \delta + \phi - (\beta \delta) - 
        (\beta \phi) - (\delta \phi) + (\beta \delta \phi)$

        $\rightarrow$ $i < 
        c + d + \delta - (\beta \delta) - (\delta \phi) + (\beta \delta \phi)$

        As seen by using a Venn Diagram or other set intuition, $\delta - 
        (\beta \delta) - (\delta \phi) + (\beta \delta \phi)$, 
        represents the number of terminals in $\delta$ not in $\beta$
        and not in $\phi$.

        The lowest case for the right hand side of the inequality is where 
        this is 0, ie, all of the terms in $\delta$ are in $\beta$ or 
        $\phi$. In this case, the inequality becomes:

        $\rightarrow$ $i < c + d$

        Which is true as long as $c$ and $d$ exist in $B$.

        Want to show $c$ and $d$ always exists in $B$:

        Suppose not, then we can rewrite $B$ without $c$ or $d$:

        $B := [\delta, -i]$

        Notice that $\delta$ is simply a fixed, yet arbitrary set of terms
        following the rules of a valid clause.

        As long as $\delta$ contains at least two terms, we can simply use 
        any two terms from $\delta$ as c and d.

        If $\delta$ does not contain two terms, then the largest possible
        length of B is 2.
        
        By Lemma 5.10, the largest clause this can imply is of length 
        $(|A| + |B| - 2)$ where $|A|$ is the length of $A$ and $|B|$ is the length 
        of $B$.

        This means the largest $E$ implied by $A$ and $B$ has the same 
        length as $A$.

        Because the lengths are equal, such a case would not allow $A$ to 
        have length $k - 1$ and $E$ to have length $k$.

        Such a case would be inconsistent with the conditions of this 
        lemma and would not apply.
        
        Therefore, at least two terms have to exist in \{$c, d$\}.

        Since $c$ and $d$ always exist in $B$, you can derive $D$ by processing
        clauses of length at most $k-1$.
    
        Consider the case where the length of $D$ is $k - 1$:

        This means $k$ is one greater than the length of $D$, so $k$ is now:

        $k = b + c + d + e + f + \beta + \delta + \phi - (\beta \delta) - 
        (\beta \phi) - (\delta \phi) + (\beta \delta \phi) + 1$

        Similarly as before, the inequality will become:

        $\rightarrow$ $i < c + d + 1$

        Which is again true as long as $c$ and $d$ are in $B$ and $i$ is in $A$.

        As seen in the first half of this proof, $c$ and $d$ have to be in
        $B$.

        $i$ has to be in $A$ by the conditions of this lemma.

        Since the length of $F$ is less than $k$ in all cases, you can derive
        $D$ by processing only clauses with a maximum length of $k-1$.
    \end{proof}

    \begin{lemma}
        Given the following:
        \begin{itemize}
            \item $A$ is a clause of length less than $k$
            \item $B$ is a clause of length $k$
            \item $C$ is a clause of length less than $k$
            \item $D$ is a clause of length $k$ or $k - 1$
            \item $A$ expands to imply $B$ by Lemma 5.8
            \item $B$ and $C$ imply $D$ by Lemma 5.9
        \end{itemize}
        Then $A$ and $C$ can imply $D$ by processing clauses of at 
        most length $k - 1$.
    \end{lemma}
    \begin{proof}

        Consider the following figure

        \begin{figure}[h]
            \includegraphics[scale=0.8]{318}
            \caption{An illustration of the derivations described in the lemma}
            \Description[An illustration of the derivations described in the lemma]{An illustration of the derivations described in the lemma}
        \end{figure}

        Let the clauses be defined with the following generic definitions:

        $A := [a, b, \beta]$
        
        $A' := [a, b, \beta, c]$

        $B := [a, b, \beta, c, d, \delta]$

        $C_1$ := [-a, e, f, $\phi$]

        $C_2$ := [-c, e, f, $\phi$]

        $D_1$ := [b, $\beta$, c, d, $\delta$, e, f, $\phi$]

        $D_2$ := [a, b, $\beta$, d, $\delta$, e, f, $\phi$]

        E := [b, $\beta$, e, f, $\phi$]

        F := [a, b, $\beta$, e, f, $\phi$]

        Where $\beta, \delta$, and $\phi$ are fixed, yet arbitrary sets of terms 
        abiding by the rules for a valid clause (see Lemma 5.3) and consistent
        with the conditions of this lemma.

        Notice that $C$ has to contain a term from $B$ in the opposite form by 
        lemma 5.9.

        Notice that $B$ is made up of terms from $A$ and terms not in $A$
        by lemma 5.6.

        The term in $C$ which is opposite from the term in $B$ can therefore
        be opposite (1) from a term
        in $A$ (in this case, use $C_1$ and $D_1$) or (2) a 
        term not in $A$ (in this case use $C_2$ and $D_2$).
        
        (1) Consider the opposite form term in $C$ is in $A$, ie, use $C_1$
        and $D_1$:

        Notice $A$ and $C_1$ share an opposite term, then they can derive
        a clause, $E$, by Lemma 5.9. And $E$ can imply a clause $D_1$ by Lemma 5.8:

        \begin{figure}[h]
            \includegraphics[scale=0.8]{318b.png}
            \caption{An illustration of the derivations by clauses $A$, $C_1$, and $E$}
            \Description[An illustration of the derivations by clauses $A$, $C_1$, and $E$]{An illustration of the derivations by clauses $A$, $C_1$, and $E$}
        \end{figure}

        Want to show you only have to process clauses whose length is
        less than $k$ to derive $D_1$.

        Want to show length of $E$ is less than $k$.

        Consider two cases: (1a) $D$ is of length $k$ and 
        (1b) $D$ is of length $k - 1$
        
        Consider (1a) where $D$ is of length $k$:

        In the following equations, let the presence of a term represent a count of one,
        and the presence of a set of terms represent the number of terms in that set. If
        multiple sets of terms are shown in parentheses, let this represent the number
        of terms found in both sets.

        Since $D$ is of length $k$, we can define $k$ as follows:

        $k = b + c + d + e + f + \beta + \delta + \phi - (\beta \delta) 
        - (\beta \phi) - (\delta \phi) + (\beta \delta \phi)$

        Length of $E = b + e + f + \beta + \phi - (\beta \phi)$
        
        Want to show the length of $E$ is less than $k$:

        $b + e + f + \beta + \phi - (\beta \phi) < b + c + d + e + f + 
        \beta + \delta + \phi - (\beta \delta) - (\beta \phi) - (\delta \phi) 
        + (\beta \delta \phi)$

        $\rightarrow 0 < c + d + \delta - (\beta \delta) 
        - (\delta \phi) + (\beta \delta \phi)$

        Using a Venn Diagram or by other set intuition, $\delta - 
        (\beta \delta) - (\delta \phi) + (\beta \delta \phi)$ represents
        the number of terms in $\delta$ not in $\beta$ and not in $\phi$. 
        The lowest this value can be is 0 if all terms in $\delta$ are in 
        $\beta$ or $\phi$.

        The inequality becomes:

        $\rightarrow 0 < c + d$

        Which is true as long as $c$ or $d$ exist in $B$.

        Want to show $c$ or $d$ always exists in $B$.

        Suppose not, then we can rewrite $B$ as follows:

        $B := [a, b, \beta]$

        Note that the maximum length of $\delta$ is 0 because if it were 1
        or more, then we can just extract $c$ or $d$ from $\delta$.

        In this case, the length of $B$ is the same as the length of $A$.
        
        Recall the conditions of the lemma defined the length of $B$ as $k$
        and the length of $A$ as $k-1$.

        Therefore it is impossible for the lemma conditions defining the lengths
        of $A$ and $B$ to be true and $c$ or $d$ must always exist in $B$.

        Since $c$ or $d$ must exist in $B$, the inequality is always true, 
        and the length of $E$ is indeed less than $k$.

        Consider (1b) $D$ is of length $k - 1$:

        Then the length of $k$ is redefined as:

        $k = b + c + d + e + f + \beta + \delta + \phi - (\beta \delta) 
        - (\beta \phi) - (\delta \phi) + (\beta \delta \phi) + 1$

        Similarly as before, the inequality becomes:

        $\rightarrow 0 < c + d + 1$

        Which is always true so the length of $E$ is indeed less than $k$.

        Notice that, by lemma 5.8, you only have to process clauses of length
        at most $k-1$ to derive a clause of length $k$. In context, you 
        can derive $D$ via expansion while only processing clauses with
        a maximum length of $k-1$.

        (2) Consider the case using $C_2$ and $D_2$:

        We can construct a clause, $A'$, such that $A$ expands to $A'$ by 
        Lemma 5.8 and there is a term in $A'$ whose opposite term is in $C_2$.

        By Lemma 5.9, $A'$ and $C_2$ imply a clause, $F$.

        And by Lemma 5.6, all the terms in $F$ are in $D_2$ so $F$ implies $D_2$.

        This is seen in the following figure:

        \begin{figure}[h]
            \includegraphics[scale=0.8]{318c.png}
            \caption{An illustration of the derivations using $A', C_2$, and $F$}
            \Description[An illustration of the derivations using $A', C_2$, and $F$]{An illustration of the derivations using $A', C_2$, and $F$}
        \end{figure}
        
        Want to show (2a) $A'$  is shorter than $k$ and (2b) $F$ is shorter than $k$.

        (2a) Want to show the length of $A'$ is less than $k$:

        Consider two cases, (2ai) $D_2$ is of length $k$ and 
        (2aii) $D_2$ is of length k - 1

        Consider (2ai) $D_2$ is of length $k$, then we can define k as follows:

        $k = b + c + d + e + f + \beta + \delta + \phi - (\beta \delta) 
        - (\beta \phi) - (\delta \phi) + (\beta \delta \phi)$

        length of $A' = a + b + \beta + c$

        Want to show the length of $A'$ is less than $k$:

        $\rightarrow$ $a + b + \beta + c$ $<$ $b + c + d + e + f + 
        \beta + \delta + \phi - (\beta \delta) 
        - (\beta \phi) - (\delta \phi) + (\beta \delta \phi)$

        $\rightarrow$ $a$ $<$ $d + e + f + \delta + \phi - (\beta \delta) 
        - (\beta \phi) - (\delta \phi) + (\beta \delta \phi)$

        By Venn Diagram or other set intuition, 
        $\delta + \phi - (\beta \delta) - (\beta \phi) - (\delta \phi) + 
        (\beta \delta \phi)$ represents the number of unique terms in 
        $\delta$ and $\phi$.
        The smallest value for this is when $\delta = \phi$, so this can
        be replaced with $\phi$.

        $\rightarrow$ $a$ $<$ $d + e + f + \phi$

        Which is clearly true as long as there are at least two terms in $d, e, f, \delta$ or $\phi$.

        Consider the cases where 0 terms exist in $d, e, f, \delta, \phi$

        Then 
        
        $D_2 := [a, b, \beta]$

        Recall 

        $A := [a, b, \beta]$

        It is seen $A$ and $D_2$ are the same length, but the conditions of the lemma state
        $A$ is of length less than $k$ and $D$ is of length $k$. 

        This is a contradiction so at least one term has to exist in $d, e, f, \delta, \phi$

        Consider the cases where 1 term exists in $d, e, f, \delta, \phi$

        Then 
        
        $D_2 := [a, b, \beta, x]$

        Where $x$ is a single term from $\{d, e, f, \delta, \phi\}$
        
        Recall $A = [a, b, \beta]$

        Since $x$ is exactly one term, $D_2$ is larger than $A$.

        The clause $A$ can expand to $D_2$ by Lemma 5.6. In this case, 
        we can expand $A$ to $D_2$ directly and we do not necessarily 
        need $A'$,
        
        Since $D_2$ is of length $k$, it can be expanded to by processing
        only clauses with a maximum length of $k-1$.

        Therefore, either the inequality holds, or if it doesn't we can
        derive $D_2$ directly without processing clauses longer than length $k-1$.

        Consider (2aii) $D_2$ is of length $k - 1$:

        $k$ is now defined as:

        $k = b + c + d + e + f + \beta + \delta + \phi - (\beta \delta) 
        - (\beta \phi) - (\delta \phi) + (\beta \delta \phi) + 1$

        Similarly as before, the inequality becomes:

        $\rightarrow$ $a$ $<$ $d + e + f + \phi + 1$

        Where $\phi$ = $\delta$

        Which is true as long as there is at least one term in
        $d, e, f, \delta$ or $\phi$.

        As seen before, if there are fewer than two terms in
        $\{d, e, f, \delta, \phi\}$ then we can derive $D_2$ without
        processing clauses larger than length $k-1$.

        Therefore if $D_2$ is of length $k - 1$, then either the length of $A'$ is 
        shorter than $k$ or $D_2$ can be derived directly from $A$ whose length is less than $k$.
        
        (2b) want to show $F$ is shorter than $k$.

        Consider two cases, (2bi) $D_2$ is of length $k$ and 
        (2bii) $D_2$ is of length $k - 1$.

        Consider (2bi) where $D_2$ is of length $k$:
        
        By $D_2$, $k$ is defined as 

        $k = b + c + d + e + f + \beta + \delta + \phi - (\beta \delta) 
        - (\beta \phi) - (\delta \phi) + (\beta \delta \phi)$

        length of $F = a + b + e + f + \beta + \phi - (\beta \phi)$
        
        Want to show the length of $F$ is less than $k$:

        $a + b + e + f + \beta + \phi - (\beta \phi)$ $<$ 
        $b + c + d + e + f + \beta + \delta + \phi - (\beta \delta) 
        - (\beta \phi) - (\delta \phi) + (\beta \delta \phi)$

        $\rightarrow$ $a$ $<$ $c + d + \delta - (\beta \delta) - 
        (\delta \phi) + (\beta \delta \phi)$

        Using a Venn Diagram or by other set intuition, 
        $\delta - (\beta \delta) - (\delta \phi) + (\beta \delta \phi)$
        represents the terms in $\delta$ that are not in $\beta$ or $\phi$.
        
        This is lowest when all the terms in $\delta$ are in $\beta$ or $\phi$, 
        this part of the inequality then becomes 0.

        Then the inequality becomes:

        $\rightarrow$ $a$ $<$ $c + d$

        Which is clearly true if $c$ and $d$ exist.

        Consider the case where $c$ and $d$ don't exist.

        Then $B$ can be redefined:

        $B := [a, b, \beta, x]$

        where $x$ is a single term. Note that if $x$ is more than one term,
        then we could use the two terms in $x$ as $c$ and $d$.

        Since $x$ is a single term, the length of $B$ is the same as the length of $A'$ 
        which was already shown to be shorter than $k$ or irrelevant
        as $D_2$ could be derived directly from $A$ without processing
        clauses of length greater than $k-1$.

        Consider (2bii) where $D_2$ is of length $k - 1$.

        Then $k$ is defined as:

        $k = b + c + d + e + f + \beta + \delta + \phi - (\beta \delta) 
        - (\beta \phi) - (\delta \phi) + (\beta \delta \phi) + 1$

        Want to show length of $F$ is less than $k$:

        $a + b + e + f + \beta + \phi - (\beta \phi)$ $<$ 
        $b + c + d + e + f + \beta + \delta + \phi - (\beta \delta) 
        - (\beta \phi) - (\delta \phi) + (\beta \delta \phi) + 1$

        Similarly as before, this becomes:

        $\rightarrow$ $a$ $<$ $c + d + 1$

        Which is always true since we already proved $c$ and $d$ have to exist
        or if they don't we can derive $D_2$ without processing clauses
        greater than length $k-1$. 

        Since the cases where we cannot directly derive $D_2$ by processing
        clauses with a maximum length of $k-1$ imply the inequality holds, then
        $F$ is smaller than $k$ and $D_2$ has a maximum length of $k$, 
        and by Lemma 3.6 the largest a clause has to be in order to imply
        another clause of length $k$ is $k-1$. Meaning you only have to process
        clauses of length $k-1$ in order to derive a clause of length $k$.
        
        Since $A, C_1, C_2, E,$ and $F$ are all shorter than $k$ or
        $D$ can be directly derived from clauses shorter than $k$ 
        and $D$ can
        be derived by expanding $E$ or $F$, then $D$ can be derived 
        without the need to process clauses of length greater than $k - 1$.
    \end{proof}

    \begin{lemma}
        Given an instance of 3SAT, you can expand all of the given clauses
        to the point where you are considering clauses of length $n$.
    \end{lemma}
    \begin{proof}
        Given an instance of 3SAT, we know all clauses are of length 3.

        If we want to consider a generic $n$-terminal clause, $B$, that's implied by a
        given clause, $A$, then by Lemma 5.6 we know it's implied iff $B$ 
        contains all of the terms in $A$.
    \end{proof}

    \begin{lemma}
        If you expand given 3-t clauses as described in Lemma 5.13, you will
        derive $2^n$ unique clauses of length $n$ iff the instance is unsatisfiable.
    \end{lemma}
    \begin{proof}
        Want to show an unsatifiable instance $\implies$ $2^n$ unique n-terminal
        clauses can be derived from the given 3-t clauses:

        By lemma 5.4, a clause of length $n$ blocks 1 assignment. 

        Recall an instance is unsatisfiable iff all $2^n$ assignments are
        blocked.

        If a 3-terminal clause blocks an assignment, then it also implies
        the corresponding n-terminal assignment because there is one possible
        n-terminal clause for any assignment.

        Since all assignments are blocked, there exists a 3-terminal clause for each
        assignment such that the terminals in the clause are assigned to make
        the clause evalute to false.

        Similarly, for each assignment, there exists an n-terminal clause such
        that the terminals in the clause are assigned to make the clause evaluate
        to false.

        This will create a unique n-terminal clause for each assignment since
        each clause can block only one assignment and overlap would imply
        the same terminal having two values in the same assignment.

        Notice that for each of these n-terminal clauses, they must contain
        three terms from at least one given clause. If they didn't, then
        the assignment blocked by that n-terminal clause would not be blocked
        and the instance would be satisfiable.

        Since each n-terminal clause blocks one assignment, blocking all assignments
        requires $2^n$ n-terminal clauses.

        Want to show $2^n$ unique n-terminal clauses are derived by the given
        3-t clauses $\implies$ then the instance is unsatifiable.

        By lemma 5.4, a clause of length $n$ blocks 1 assignment. 

        Therefore if there are $2^n$ unique n-terminal clauses, then
        all $2^n$ assignments will be blocked.

        Note that there will be no overlap because each n-terminal clause
        sets the value for each terminal and overlap would imply the same
        terminal having two values by the same assignment which is impossible.
    \end{proof}

    \begin{lemma}
        The n-terminal clauses described in Lemma 5.14 can be reduced to derive
        any pair of contradicting 1-terminal clauses.
    \end{lemma}
    \begin{proof}
        Given $2^n$ n-terminal clauses, want to show you can imply any pair
        of contradicting 1-terminal clauses by lemma 5.7.

        Pick a terminal that will not exist in the final 1-terminal clauses.

        Half of the existing n-terminal clauses have that terminal assigned
        the value of False and the other half have that terminal assigned
        the value of True.

        Pick one clause that blocks an assignment where the terminal is True.

        Then there exists an assignment for each possible
        value for the remaining n-1 terminals.

        Therefore, there must exist another clause that shares all of the same terms, but
        where that one terminal is assigned the value of False.

        Using these two terms, we can create a new clause by lemma 5.7.

        Now all of the clauses of length n - 1 do not contain that terminal.

        Repeat this process while never selecting the same terminal twice
        until you are left with two contradicting 1-terminal clauses.
    \end{proof}

    \begin{lemma}
        Contradicting 1-terminal clauses can be expanded to imply
        every possible clause.
    \end{lemma}
    \begin{proof}
        Let the following clauses be a pair of contradicting 1-t clauses:

        [a]

        [-a]

        By lemma 5.6, we can expand to any clause that contains $a$ or $-a$.

        Let the following be a generic 3-terminal clause that does not contain
        $a$ or $-a$:

        [b, c, d, ...]

        By Lemma 5.6, we know the 1-terminal clauses imply the following 3-t clauses:

        [a, b]

        [-a, c, d, ...]

        By Lemma 5.9, these clauses imply:

        [b, c, d, ...]

        Therefore we can imply any clause containing $a, -a$, or neither, 
        which encompasses every possible clause.
    \end{proof}

    \begin{lemma}
        Given the following:
        \begin{itemize}
            \item A, B, C, and D, are clauses shorter than k
            \item E and F are clauses of length k
            \item G is a clause of length k or k - 1
            \item A and B imply E by Lemma 5.9
            \item C and D imply F by Lemma 5.9
            \item E and F imply G by Lemma 5.9
        \end{itemize}
        Then G can be implied by processing clauses with a maximum length of k - 1
    \end{lemma}
    \begin{proof}
        Consider the following figure:

        \begin{figure}[h]
            \includegraphics[scale=0.8]{517a.png}
            \caption{An illustration of the derivations described in the lemma}
            \Description[An illustration of the derivations described in the lemma]{An illustration of the derivations described in the lemma}
        \end{figure}

        Let the clauses be defined as follows:

        $A := [a, b, \beta, i]$

        $B := [c, d, \delta, -i]$
        
        $C := [-a, f, \phi, j]$
        
        $D := [g, h, \gamma, -j]$

        Then the following are implied by Lemma 5.9:

        $E = [a, b, \beta, c, d, \delta]$
        
        $F = [-a, f, \phi, g, h, \gamma]$
        
        $G = [b, \beta, c, d, \delta, f, \phi, g, h, \gamma]$
        
        Where $\beta, \delta, \phi,$ and $\gamma$ are fixed yet arbitrary sets
        of terms consistent with the rules for a valid clause as by Lemma 5.3.

        Notice that $E$ and $F$ have to share a term of the opposite form
        in order to imply $G$ by Lemma 5.9. All of the terms in $E$ and $F$
        came from $A, B, C,$ and $D$.
        Since $A, B, C,$ and $D$ are
        all arbitrary clauses, selecting which term to negate does not
        have an effect on the outcome as long as one form of the term exists in $E$
        and the other form exists in $F$.
        Here the opposite term is shared 
        between $A$ and $D$, but any term that appears positive in $A$
        or $B$ and negated in $C$ or $D$ will yield the same results.
 
        Want to show $G$ can be derived by processing clauses with a 
        maximum length of k - 1.

        Define additional implications:

        $H = [b, \beta, i, f, \phi, j]$ (By clauses $A$ and $C$)

        $I = [c, d, \delta, b, \beta, f, \phi, j]$ (By clauses $B$ and $H$)

        $J = [g, h, \gamma, c, d, \delta, b, \beta, f, \phi]$ (By clauses $D$ and $I$)

        Notice $J$ is equivalent to $G$.

        Want to show (1) $H$ is shorter than $k$ and (2) $I$ is shorter than $k$

        (1) Want to show $H$ is shorter than $k$

        Two cases to consider (1a) $G$ is of length $k$ and (1b) $G$ is of length $k - 1$

        (1a) $G$ is of length $k$

        In the following equations, let the presence of a term represent a count of one,
        and the presence of a set of terms represent the number of terms in that set. If
        multiple sets of terms are shown in parentheses, let this represent the number
        of terms found in both sets.

        Since $G$ is of length $k$ we can define $k$ as follows:

        $k = b + c + d + f + g + h
            + \beta + \delta + \phi + \gamma
            - (\beta \delta) - (\beta \phi) - (\beta \gamma) - (\delta \phi) - (\delta \gamma) -(\phi \gamma)
            + (\beta \delta \phi) + (\beta \delta \gamma) + (\beta \phi \gamma) + (\delta \phi \gamma)
            - (\beta \delta \phi \gamma)
        $

        Length of $H = b + i + f + j + \beta + \phi - (\beta \phi)$
        
        Want to show the length of $H$ is less than $k$:

        $b + i + f + j + \beta + \phi - (\beta \phi)$
        $<$
        $b + c + d + f + g + h
            + \beta + \delta + \phi + \gamma
            - (\beta \delta) - (\beta \phi) - (\beta \gamma) - (\delta \phi) - (\delta \gamma) -(\phi \gamma)
            + (\beta \delta \phi) + (\beta \delta \gamma) + (\beta \phi \gamma) + (\delta \phi \gamma)
            - (\beta \delta \phi \gamma)
        $

        $\rightarrow$
        $i + j$
        $<$
        $c + d + g + h
            + \delta + \gamma
            - (\beta \delta) - (\beta \gamma) - (\delta \phi) - (\delta \gamma) -(\phi \gamma)
            + (\beta \delta \phi) + (\beta \delta \gamma) + (\beta \phi \gamma) + (\delta \phi \gamma)
            - (\beta \delta \phi \gamma)
        $

        % The R.H.S. has the following set intuition:

        % \begin{center}
        %     \begin{tabular}{ |c|c|c|c|c|c|c|c|c|c|c|c|c|c|c|c| }
        %         \hline
        %         term & $\beta$ & $\delta$ & $\phi$ & $\gamma$ & 
        %         $\beta \delta$ & $\beta \phi$ & $\beta \gamma$ & 
        %         $\delta \phi$ & $\delta \gamma$ & $\phi \gamma$ &
        %         $\beta \delta \phi$ & $\beta \delta \gamma$ &
        %         $\beta \phi \gamma$ & $\delta \phi \gamma$ &
        %         $\beta \delta \phi \gamma$\\
        %         \hline
        %         $ + \delta$ & 0 & +1 & 0 & 0 & +1 & 0 & 0 & +1 & +1 & 0 & +1 & +1 & 0 & +1 & +1 \\
        %         \hline
        %         $ + \gamma$ & 0 & 0 & 0 & +1 & 0 & 0 & +1 & 0 & +1 & +1 & 0 & +1 & +1 & +1 & +1 \\
        %         \hline
        %         $- (\beta \delta)$ & 0 & 0 & 0 & 0 & -1 & 0 & 0 & 0 & 0 & 0 & -1 & -1 & 0 & 0 & -1 \\
        %         \hline
        %         $- (\beta \gamma)$ & 0 & 0 & 0 & 0 & 0 & 0 & -1 & 0 & 0 & 0 & 0 & -1 & -1 & 0 & -1 \\
        %         \hline
        %         $- (\delta \phi)$ & 0 & 0 & 0 & 0 & 0 & 0 & 0 & -1 & 0 & 0 & -1 & 0 & 0 & -1 & -1 \\
        %         \hline
        %         $- (\delta \gamma)$ & 0 & 0 & 0 & 0 & 0 & 0 & 0 & 0 & -1 & 0 & 0 & -1 & 0 & -1 & -1 \\
        %         \hline
        %         $- (\phi \gamma)$ & 0 & 0 & 0 & 0 & 0 & 0 & 0 & 0 & 0 & -1 & 0 & 0 & -1 & -1 & -1 \\
        %         \hline
        %         $+ (\beta \delta \phi)$ & 0 & 0 & 0 & 0 & 0 & 0 & 0 & 0 & 0 & 0 & +1 & 0 & 0 & 0 & +1 \\
        %         \hline
        %         $+ (\beta \delta \gamma)$ & 0 & 0 & 0 & 0 & 0 & 0 & 0 & 0 & 0 & 0 & 0 & +1 & 0 & 0 & +1 \\
        %         \hline
        %         $+ (\beta \phi \gamma)$ & 0 & 0 & 0 & 0 & 0 & 0 & 0 & 0 & 0 & 0 & 0 & 0 & +1 & 0 & +1 \\
        %         \hline
        %         $+ (\delta \phi \gamma)$ & 0 & 0 & 0 & 0 & 0 & 0 & 0 & 0 & 0 & 0 & 0 & 0 & 0 & +1 & +1 \\
        %         \hline
        %         $- (\beta \delta \phi \gamma)$ & 0 & 0 & 0 & 0 & 0 & 0 & 0 & 0 & 0 & 0 & 0 & 0 & 0 & 0 & -1 \\
        %         \hline
        %         total & 0 & 1 & 0 & 1 & 0 & 0 & 0 & 0 & 1 & 0 & 0 & 0 & 0 & 0 & 0 \\
        %         \hline
        %     \end{tabular}
        % \end{center}

        Using a Venn Diagram or by other set intuition, the generic sets of 
        terms on the R.H.S. of the inequality represent the following:

        \begin{itemize}
            \item terms just in $\delta$
            \item terms just in $\gamma$
            \item terms in both $\delta$ and $\gamma$
        \end{itemize}
        
        The lowest possible value for this is 0 when the three aforementioned sets
        have no terms.

        The inequality becomes:

        $b + i + f + j < b + c + d + f + g + h$

        $\rightarrow$ $i + j < c + d + g + h$

        Which requires at least three unique terms in $c, d, g,$ and $h$.

        Want to show at least three unique terms have to exist in 
        $c, d, g,$ and $h$.

        Suppose not. Then at most two unique terms exist in $c, d, g,$ and $h$.

        Want to show the length of either E or F is the same as A, B, C, or D.

        Consider all cases where there are fewer than three terms in {$c, d, g, h$}

        Case 1, 0 terms exist in {$c, d, g, h$}

        Recall the value for $B$

        $B := [c, d, \delta, -i]$

        But since 0 terms exist in {$c, d, g, h$}, this becomes

        $B := [-i]$

        Note that no terms may exist in $\delta$ because any terms
        in $\delta$ could be extracted to count as $c$ or $d$.

        Then a new derivation of $E$ occurs:

        $E := [a, b, \beta]$

        Which is identical to $A$ except for a missing $i$ terminal.

        Therefore the length of $E$ is less than the length of $A$.

        This is a contradiction since the length of $E$ is given as $k$
        and the length of $A$ is given as less than $k$.

        Therefore this case is impossible.

        Case 2, 1 term exists in {$c, d, g, h$}

        It was shown that if 0 terms exist in {$c, d$}, there's a contradiction, 
        so at least one term has to exist in {$c, d$}.

        These terms are fixed, yet arbitrary so let's pick $c$ to be the one term that exists.

        $B$ is now assigned:

        $B := [c, -i]$

        Again, no additional terms may exist in $\delta$ since they could
        be used as the terminal $d$.

        $E$ is again recalculated as:

        $E := [a, b, \beta, c]$

        By Lemma 5.10, the maximum length of E is 
        (length of $A$ + length of $B$ - 2).

        Since exactly one term exists in $c$ and exactly one term exists in $-i$,
        the length of $B$ is 2 and therefore the length of $E$ is the same
        as the length of $A$.

        However it was given that the length of $E$ is $k$ and the length of 
        $A$ is less than $k$.

        This is a contradiction so this case cannot exist.

        Case 3, 2 terms exist in {$c, d, g, h$}

        As seen in Case 2, at least two terms must exist in {$c, d$}. This
        means no terms may exist in {$g, h$}.

        The clause $D$ can be rewritten:

        $D := [-j]$

        Note that $\gamma$ cannot contain any terms because any terms in
        $\gamma$ could be extracted and used as $g$ or $h$

        Since $-j$ represents one terminal, the length of $D$ is 1.

        Now the maximum length of $F$ is (length of $C$ + length of $D$ - 2).
        Which means the maximum length of $F$ is one less than the length of $C$.

        However it was given that the length of $F$ is $k$ and the length
        of $C$ is less than $k$.

        This is a contradiction so this case cannot exist.

        Since at least three terms must exist in {$c, d, g, h$}, the
        inequality holds and $H$ is shorter than $k$.

        (1b) $G$ is of length $k - 1$

        If $G$ is of length $k - 1$, then k is defined as:

        $k = b + c + d + f + g + h
            + \beta + \delta + \phi + \gamma
            - (\beta \delta) - (\beta \phi) - (\beta \gamma) - (\delta \phi) - (\delta \gamma) -(\phi \gamma)
            + (\beta \delta \phi) + (\beta \delta \gamma) + (\beta \phi \gamma) + (\delta \phi \gamma)
            - (\beta \delta \phi \gamma)
            + 1
        $

        Similarly to before, the inequality becomes:

        $\rightarrow$ $i + j < c + d + g + h + 1$

        Which is true as long as at least two terms exist in {$c, d, g, h$}.

        We already showed at least three terms must exist in {$c, d, g, h$}
        so the inequality is always true.

        Therefore $H$ is always shorter than $k$. 

        (2) Want to show $I$ is shorter than $k$

        Two cases to consider (2a) $G$ is of length $k$ and (2b) $G$ is of length $k - 1$

        (2a) $G$ is of length $k$

        Since $G$ is of length $k$ we can define $k$ as follows:

        $k = b + c + d + f + g + h
            + \beta + \delta + \phi + \gamma
            - (\beta \delta) - (\beta \phi) - (\beta \gamma) - (\delta \phi) - (\delta \gamma) -(\phi \gamma)
            + (\beta \delta \phi) + (\beta \delta \gamma) + (\beta \phi \gamma) + (\delta \phi \gamma)
            - (\beta \delta \phi \gamma)
        $

        Length of $I = c + d + b + f + j
        + \delta + \beta + \phi
        - (\delta \beta) - (\delta \phi) - (\beta \phi)
        + (\delta \beta \phi)
        $

        Want to show the length of $I$ is less than $k$:

        $c + d + b + f + j
        + \delta + \beta + \phi
        - (\delta \beta) - (\delta \phi) - (\beta \phi)
        + (\delta \beta \phi)
        $
        $<$
        $b + c + d + f + g + h
            + \beta + \delta + \phi + \gamma
            - (\beta \delta) - (\beta \phi) - (\beta \gamma) - (\delta \phi) - (\delta \gamma) -(\phi \gamma)
            + (\beta \delta \phi) + (\beta \delta \gamma) + (\beta \phi \gamma) + (\delta \phi \gamma)
            - (\beta \delta \phi \gamma)
        $

        By Venn Diagram or other set intuition, 
        the generic sets of terms on the L.H.S. represent the number 
        of unique terms in $\delta, \beta,$ and $\phi$. Similarly, 
        the generic sets of terms on the R.H.S. represent the number
        of unique terms in $\delta, \beta, \phi,$ and $\gamma$.

        Subtacting the shared sets of terms from both sides, on
        the R.H.S. we are left
        with the terms in $\gamma$ that are not present in 
        $\delta, \beta, or \phi$. The lowest value for this is 0 when
        all terms in $\gamma$ exist in $\delta, \beta,$ and $\phi$.

        The inequality therefore becomes:

        $c + d + b + f + j < b + c + d + f + g + h$

        $\rightarrow j < g + h$

        Which is true as long as both g and h exist.

        Want to show g and h exist.

        Suppose not, then g and h don't exist.

        Recall the value for $D$:

        $D := [g, h, \gamma, -j]$

        Since $g$ and $h$ don't exist, this becomes

        $D := [-j]$

        Note that no terms may exist in $\gamma$ because any terms in $\gamma$
        could be extracted and counted as $g$ or $h$.

        Now we recalculate $E$:

        $E := [-a, f, \phi]$

        Which is smaller than $C$ because $j$ has to exist by the lemma's conditions.

        However it was given that the length of $E$ is $k$ and the length of $C$
        is less than $k$.

        This is a contradiction, therefore $g$ and $h$ must exist.

        Therefore the inequality is always true and the length of $I$ is less than $k$.

        (2b) $G$ is of length $k - 1$

        Similarly as before, we define k in terms of the length of G:

        $k = b + c + d + f + g + h
            + \beta + \delta + \phi + \gamma
            - (\beta \delta) - (\beta \phi) - (\beta \gamma) - (\delta \phi) - (\delta \gamma) -(\phi \gamma)
            + (\beta \delta \phi) + (\beta \delta \gamma) + (\beta \phi \gamma) + (\delta \phi \gamma)
            - (\beta \delta \phi \gamma)
            + 1
        $

        Similarly to before, the inequality becomes
        
        $\rightarrow j < g + h + 1$

        Which is true as long as $g$ and $h$ exist. 

        We already showed $g$ and $h$ exist so the inequality always 
        holds true.

        Therefore, the length of $I$ is at most $k - 1$.

        Since we can derive J (which is equivalent to G) by processing
        A, B, C, D, H, and I, all of which whose lengths are shorter than k, 
        we can derive G by only processing clauses whose lengths are 
        less than k.
    \end{proof}

    \begin{lemma}
        Given the following:
        \begin{itemize}
            \item a clause, $A$, of length less than $k$
            \item a clause, $B$, of length less than $k$
            \item a clause, $C$, of length less than $k$
            \item a clause, $D$, of length $k$
            \item a clause $E$, of length $k$
            \item a clause $F$, of length $k - 1$ or $k$
            \item A and B imply D by Lemma 5.9
            \item C expands to E by Lemma 5.8
            \item D and E imply F by Lemma 5.9
        \end{itemize}
        Then F can be implied by only processing clauses with a maximum
        length of $k-1$.
    \end{lemma}
    \begin{proof}
        Consider the following figure:

        \begin{figure}[h]
            \includegraphics[scale=0.8]{518a}
            \caption{An illustration of the derivations described by the lemma}
            \Description[An illustration of the derivations described by the lemma]{An illustration of the derivations described by the lemma}
        \end{figure}

        Let the following clauses be defined:

        $A := [a, b, \beta, i]$

        $B := [c, d, \delta, -i]$

        $C_1 := [-a, f, \phi]$
        
        $C_2 := [e, f, \phi]$
        
        Then the following clauses are implied:
        
        $D = [a, b, \beta, c, d, \delta]$
        
        $E_1 = [-a, f, \phi, g, h, \gamma]$
        
        $E_2 = [e, f, \phi, -a, h, \gamma]$
        
        $F_1 = [b, \beta, c, d, \delta, f, \phi, g, h, \gamma]$
        
        $F_2 = [b, \beta, c, d, \delta, e, f, \phi, h, \gamma]$
        
        Where $\beta, \delta, \phi$ and $\gamma$ are generic sets of terms
        such that A, B, and C block at least one clause by Lemma 5.3.
        
        Notice E and D must share a term of the opposite form and there
        are two possible cases for what this term is: (1) it exists in
        C or (2) it exists in the additional terms in E, ie, it does not
        exist in C. Either way the opposite term must exist in A or B
        since it must exist in D and D is composed of terms from A or B.
        Since A and B are both generic clauses, it does not matter which
        clause the term exists in as long as it is fixed.

        Consider (1) the opposite form term exists in C (use $E_1$ and $F_1$)

        Want to show $F_1$ can be derived by processing clauses with a maximum
        length of k - 1.

        Let the following clauses be defined:

        $G = [b, \beta, i, f, \phi]$ (By clause A and $C_1$ with Lemma 5.9)

        $H = [b, \beta, f, \phi, c, d, \delta]$ (By clause G and B with Lemma 5.9)

        Since all of the terms in $H$ exist in $F_1$, Lemma 5.8 can be used to
        derive $F_1$ from $H$.

        Want to show (1a) $G$ is shorter than $k$ and (1b) $H$ is shorter than $k$

        (1a) $G$ is shorter than $k$.

        Consider two cases (1ai) $F_1$ is of length $k$ and (1aii) $F_1$ is of length $k-1$.

        (1ai) $F_1$ is of length k

        In the following equations, let the presence of a term represent a count of one,
        and the presence of a set of terms represent the number of terms in that set. If
        multiple sets of terms are shown in parentheses, let this represent the number
        of terms found in both sets.
        
        Recall $F_1$ is of length k so k can be defined as follows:
        
        $k = b + c + d + f + g + h 
        + \beta + \delta + \phi + \gamma
        - (\beta \delta) - (\beta \phi) - (\beta \gamma) - (\delta \phi) - (\delta \gamma) - (\phi \gamma)
        + (\beta \delta \phi) + (\beta \delta \gamma) + (\beta \phi \gamma) + (\delta \phi \gamma)
        - (\beta \delta \phi \gamma)
        $

        Length of $G = b + i + f + \beta + \phi - (\beta \phi)$

        Want to show length of G is less than k:

        $b + i + f + \beta + \phi - (\beta \phi)$
        $<$
        $b + c + d + f + g + h 
        + \beta + \delta + \phi + \gamma
        - (\beta \delta) - (\beta \phi) - (\beta \gamma) - (\delta \phi) - (\delta \gamma) - (\phi \gamma)
        + (\beta \delta \phi) + (\beta \delta \gamma) + (\beta \phi \gamma) + (\delta \phi \gamma)
        - (\beta \delta \phi \gamma) 
        $

        $\rightarrow$ $b + i + f$
        $<$
        $b + c + d + f + g + h 
         + \delta + \gamma
        - (\beta \delta) - (\beta \gamma) - (\delta \phi) - (\delta \gamma) - (\phi \gamma)
        + (\beta \delta \phi) + (\beta \delta \gamma) + (\beta \phi \gamma) + (\delta \phi \gamma)
        - (\beta \delta \phi \gamma) 
        $

        Here's a representation of the sets and how each term affects the size of a set

        \begin{center}
            \begin{tabular}{ |c|c|c|c|c|c|c|c|c|c|c|c|c|c|c|c| }
                \hline
                term & $\beta$ & $\delta$ & $\phi$ & $\gamma$ & 
                $\beta \delta$ & $\beta \phi$ & $\beta \gamma$ & 
                $\delta \phi$ & $\delta \gamma$ & $\phi \gamma$ &
                $\beta \delta \phi$ & $\beta \delta \gamma$ &
                $\beta \phi \gamma$ & $\delta \phi \gamma$ &
                $\beta \delta \phi \gamma$\\
                \hline
                $ + \delta$ & 0 & +1 & 0 & 0 & +1 & 0 & 0 & +1 & +1 & 0 & +1 & +1 & 0 & +1 & +1 \\
                \hline
                $ + \gamma$ & 0 & 0 & 0 & +1 & 0 & 0 & +1 & 0 & +1 & +1 & 0 & +1 & +1 & +1 & +1 \\
                \hline
                $- (\beta \delta)$ & 0 & 0 & 0 & 0 & -1 & 0 & 0 & 0 & 0 & 0 & -1 & -1 & 0 & 0 & -1 \\
                \hline
                $- (\beta \gamma)$ & 0 & 0 & 0 & 0 & 0 & 0 & -1 & 0 & 0 & 0 & 0 & -1 & -1 & 0 & -1 \\
                \hline
                $- (\delta \phi)$ & 0 & 0 & 0 & 0 & 0 & 0 & 0 & -1 & 0 & 0 & -1 & 0 & 0 & -1 & -1 \\
                \hline
                $- (\delta \gamma)$ & 0 & 0 & 0 & 0 & 0 & 0 & 0 & 0 & -1 & 0 & 0 & -1 & 0 & -1 & -1 \\
                \hline
                $- (\phi \gamma)$ & 0 & 0 & 0 & 0 & 0 & 0 & 0 & 0 & 0 & -1 & 0 & 0 & -1 & -1 & -1 \\
                \hline
                $+ (\beta \delta \phi)$ & 0 & 0 & 0 & 0 & 0 & 0 & 0 & 0 & 0 & 0 & +1 & 0 & 0 & 0 & +1 \\
                \hline
                $+ (\beta \delta \gamma)$ & 0 & 0 & 0 & 0 & 0 & 0 & 0 & 0 & 0 & 0 & 0 & +1 & 0 & 0 & +1 \\
                \hline
                $+ (\beta \phi \gamma)$ & 0 & 0 & 0 & 0 & 0 & 0 & 0 & 0 & 0 & 0 & 0 & 0 & +1 & 0 & +1 \\
                \hline
                $+ (\delta \phi \gamma)$ & 0 & 0 & 0 & 0 & 0 & 0 & 0 & 0 & 0 & 0 & 0 & 0 & 0 & +1 & +1 \\
                \hline
                $- (\beta \delta \phi \gamma)$ & 0 & 0 & 0 & 0 & 0 & 0 & 0 & 0 & 0 & 0 & 0 & 0 & 0 & 0 & -1 \\
                \hline
                total & 0 & 1 & 0 & 1 & 0 & 0 & 0 & 0 & 1 & 0 & 0 & 0 & 0 & 0 & 0 \\
                \hline
            \end{tabular}
        \end{center}

        It is seen the sets of generic terms on the R.H.S. represent the count of terms in $\delta$, $\gamma$, and $\delta \gamma$.

        Let's call this $y$. The inequality then becomes:

        $\rightarrow$ $i$
        $<$
        $c + d + g + h + y
        $

        which is true as long as two terms exist in {$c, d, g, h, y$}.

        Consider the two cases: zero terms exist in {$c, d, g, h, y$} or one term exists in {$c, d, g, h, y$}

        Zero terms exist in the aforementioned set:

        In this case, we may redefine some clauses:

        $B := [-i]$

        $D := [a, b, \beta]$

        Notice $B$ must consist of a single term, $-i$, because if any terms existed in $\delta$, they could be extracted and used as $c$ or $d$ but we are considering the case where $c$ and $d$ do not exist.

        Now we have 
        
        $D := [a, b, \beta]$

        $A := [a, b, \beta, i]$

        Recall we defined the length of $D$ as $k$ and the length of $A$ as less than $k$. Clearly it is seen that the length of $A$ is one greater than the length of $D$. This is impossible with the given lengths. So this case is impossible, at least one term must exist in {$c, d, g, h, y$}

        Consider the case where one term exists in the aforementioned set:

        It was previously seen that if no terms exist in {$c, d$} then we will reach a contradiction, so the one term that exists in {$c, d, g, h, y$} must exist in {$c, d$}

        Let's redefine clauses similarly to before:

        $B := [c, -i]$

        $D := [a, b, \beta, c]$

        Since $c$ and $d$ are generic terms that are treated the same throughout this example (they always exist together), it does not matter which term we pick. So let's pick $c$ to be the term that exists. Again, we can have no terms in $\delta$ since any additional term could be extracted and treated as $d$ which we know does not exist. 

        Now we know

        $D := [a, b, \beta, c]$

        $A := [a, b, \beta, i]$

        Since $c$ and $i$ both represent a single terminal, it is seen that the length of $D$ is equal to the length of $A$. Recall the length of $D$ was defined as $k$ and the length of $A$ was defined to be less than $k$. Therefore this is a contradiction and the case where a single term exists in {$c, d, g, h, y$} is impossible.

        Since a contradiction exists in the cases where zero and one terms exist in {$c, d, g, h, y$} we know at least two terms must exist in that set. We also know that if at least two terms exist in that set, the inequality holds and the length of G is less than k.

        (1aii) F is of length k - 1

        In the following equations, let the presence of a term represent a count of one,
        and the presence of a set of terms represent the number of terms in that set. If
        multiple sets of terms are shown in parentheses, let this represent the number
        of terms found in both sets.
        
        Recall $F$ is of length k - 1 so k can be defined as follows:
        
        $k = b + c + d + f + g + h 
        + \beta + \delta + \phi + \gamma
        - (\beta \delta) - (\beta \phi) - (\beta \gamma) - (\delta \phi) - (\delta \gamma) - (\phi \gamma)
        + (\beta \delta \phi) + (\beta \delta \gamma) + (\beta \phi \gamma) + (\delta \phi \gamma)
        - (\beta \delta \phi \gamma)
        + 1
        $

        Length of $G = b + i + f + \beta + \phi - (\beta \phi)$

        Want to show length of G is less than k:

        $b + i + f + \beta + \phi - (\beta \phi)$
        $<$
        $b + c + d + f + g + h 
        + \beta + \delta + \phi + \gamma
        - (\beta \delta) - (\beta \phi) - (\beta \gamma) - (\delta \phi) - (\delta \gamma) - (\phi \gamma)
        + (\beta \delta \phi) + (\beta \delta \gamma) + (\beta \phi \gamma) + (\delta \phi \gamma)
        - (\beta \delta \phi \gamma) 
        + 1
        $

        Similarly as before, this will derive to:

        $\rightarrow$ $i$
        $<$
        $c + d + g + h + y
        + 1
        $

        Where $y$ represents the number of terms exactly in the sets $\delta$, $\gamma$, and $\delta \gamma$ (as defined in section 1ai). We already showed that at least two terms must exist in the set {$c, d, g, h, y$} so this inequality will always be true.

        (1b) Want to show $H$ is shorter than $k$.

        We have two cases to consider: (1bi) $F_1$ is of length $k$ and (1bii) $F_1$ is of length $k-1$

        Consider (1bi) $F_1$ is of length $k$

        Recall $F_1$ is of length $k$, so we can define $k$ as follows:

        $k = b + c + d + f + g + h 
        + \beta + \delta + \phi + \gamma
        - (\beta \delta) - (\beta \phi) - (\beta \gamma) - (\delta \phi) - (\delta \gamma) - (\phi \gamma)
        + (\beta \delta \phi) + (\beta \delta \gamma) + (\beta \phi \gamma) + (\delta \phi \gamma)
        - (\beta \delta \phi \gamma)
        $

        Length of $H = b + f + c + d 
        + \beta + \delta + \phi
        - (\beta \delta) - (\beta \phi) - (\delta \phi)
        + (\beta \delta \phi) $

        Want to show the length of H is less than k.

        $b + f + c + d 
        + \beta + \delta + \phi
        - (\beta \delta) - (\beta \phi) - (\delta \phi)
        + (\beta \delta \phi) $
        $<$
        $b + c + d + f + g + h 
        + \beta + \delta + \phi + \gamma
        - (\beta \delta) - (\beta \phi) - (\beta \gamma) - (\delta \phi) - (\delta \gamma) - (\phi \gamma)
        + (\beta \delta \phi) + (\beta \delta \gamma) + (\beta \phi \gamma) + (\delta \phi \gamma)
        - (\beta \delta \phi \gamma)
        $

        $\rightarrow$
        $b + f + c + d$
        $<$
        $b + c + d + f + g + h 
        + \gamma
        - (\beta \gamma) - (\delta \gamma) - (\phi \gamma)
        + (\beta \delta \gamma) + (\beta \phi \gamma) + (\delta \phi \gamma)
        - (\beta \delta \phi \gamma)
        $

        The following is set intuition for the R.H.S. of the inequality:

        \begin{center}
            \begin{tabular}{ |c|c|c|c|c|c|c|c|c|c|c|c|c|c|c|c| }
                \hline
                term & $\beta$ & $\delta$ & $\phi$ & $\gamma$ & 
                $\beta \delta$ & $\beta \phi$ & $\beta \gamma$ & 
                $\delta \phi$ & $\delta \gamma$ & $\phi \gamma$ &
                $\beta \delta \phi$ & $\beta \delta \gamma$ &
                $\beta \phi \gamma$ & $\delta \phi \gamma$ &
                $\beta \delta \phi \gamma$\\
                \hline
                $ + \gamma$ & 0 & 0 & 0 & +1 & 0 & 0 & +1 & 0 & +1 & +1 & 0 & +1 & +1 & +1 & +1 \\
                \hline
                $- (\beta \gamma)$ & 0 & 0 & 0 & 0 & 0 & 0 & -1 & 0 & 0 & 0 & 0 & -1 & -1 & 0 & -1 \\
                \hline
                $- (\delta \gamma)$ & 0 & 0 & 0 & 0 & 0 & 0 & 0 & 0 & -1 & 0 & 0 & -1 & 0 & -1 & -1 \\
                \hline
                $- (\phi \gamma)$ & 0 & 0 & 0 & 0 & 0 & 0 & 0 & 0 & 0 & -1 & 0 & 0 & -1 & -1 & -1 \\
                \hline
                $+ (\beta \delta \gamma)$ & 0 & 0 & 0 & 0 & 0 & 0 & 0 & 0 & 0 & 0 & 0 & +1 & 0 & 0 & +1 \\
                \hline
                $+ (\beta \phi \gamma)$ & 0 & 0 & 0 & 0 & 0 & 0 & 0 & 0 & 0 & 0 & 0 & 0 & +1 & 0 & +1 \\
                \hline
                $+ (\delta \phi \gamma)$ & 0 & 0 & 0 & 0 & 0 & 0 & 0 & 0 & 0 & 0 & 0 & 0 & 0 & +1 & +1 \\
                \hline
                $- (\beta \delta \phi \gamma)$ & 0 & 0 & 0 & 0 & 0 & 0 & 0 & 0 & 0 & 0 & 0 & 0 & 0 & 0 & -1 \\
                \hline
                total & 0 & 0 & 0 & 1 & 0 & 0 & 0 & 0 & 0 & 0 & 0 & 0 & 0 & 0 & 0 \\
                \hline
            \end{tabular}
        \end{center}

        This shows the generic sets on the R.H.S. consist of the terms that
        are in $\gamma$ and in no other set. 

        Let's represent this as $y$ (we cannot just call it $\gamma$ because there could be a term in $\gamma$ that exists in another set that would not be in $y$)

        The inequality therefore becomes:

        $\rightarrow$
        $b + f + c + d$
        $<$
        $b + c + d + f + g + h 
        + y
        $

        $\rightarrow$
        $0 < g + h + y$

        Which is true if at least one term exists in {$g, h, y$}

        Suppose not. Then no terms exist in {$g, h, y$} and we can redefine some clauses:

        $E_1 := [-a, f, \phi]$

        Notice that no terms may exist in $\gamma$ because if at least one term exists in $\gamma$, it may be extracted and used as $g$ or $h$, but we said no terms may exist in {$g, h$}.

        Now we have the clauses:

        $C_1 := [-a, f, \phi]$

        $E_2 := [-a, f, \phi]$

        Notice these two clauses are identical and therefore have the same length. Recall we defined the length of $E$ as $k$ and the length of $C$ as less than $k$ so this is a contradiction. Therefore at least on term must exist in {$g, h, y$}.

        Since at least one term must exist in {$g, h, y$}, the inequality holds.

        Therefore the length of $H$ is less than $k$ when the length of $F_1$ is $k$.

        Consider (1bii) $F_1$ is of length $k-1$

        Recall $F_1$ is of length $k - 1$, so we can define $k$ as follows:

        $k = b + c + d + f + g + h 
        + \beta + \delta + \phi + \gamma
        - (\beta \delta) - (\beta \phi) - (\beta \gamma) - (\delta \phi) - (\delta \gamma) - (\phi \gamma)
        + (\beta \delta \phi) + (\beta \delta \gamma) + (\beta \phi \gamma) + (\delta \phi \gamma)
        - (\beta \delta \phi \gamma)
        + 1
        $

        Length of $H = b + f + c + d 
        + \beta + \delta + \phi
        - (\beta \delta) - (\beta \phi) - (\delta \phi)
        + (\beta \delta \phi) $

        Want to show the length of H is less than k.

        $b + f + c + d 
        + \beta + \delta + \phi
        - (\beta \delta) - (\beta \phi) - (\delta \phi)
        + (\beta \delta \phi) $
        $<$
        $b + c + d + f + g + h 
        + \beta + \delta + \phi + \gamma
        - (\beta \delta) - (\beta \phi) - (\beta \gamma) - (\delta \phi) - (\delta \gamma) - (\phi \gamma)
        + (\beta \delta \phi) + (\beta \delta \gamma) + (\beta \phi \gamma) + (\delta \phi \gamma)
        - (\beta \delta \phi \gamma)
        + 1
        $

        Similarly to before, this will derive 

        $\rightarrow$
        $0 < g + h + y + 1$

        Where $y$ represents the number of terms in $\gamma$ that exist in no other set. This is always true.
        
        Therefore the length of $H$ is less than $k$ when the length of $F_1$ is $k - 1$.

        Consider (2) the opposite form term does not exist in C.

        Recall we have the clauses:

        $C_2 := [e, f, \phi]$

        $F_2 := [b, \beta, c, d, \delta, e, f, \phi, h, \gamma]$

        Since all of the terms in $C_2$ exist in $F_2$, we can keep expanding $C_2$ by Lemma 5.7 until we derive $F_2$ and by the nature of the lemma, we will only have to process a clause of length 1 shorter than $F_2$ to derive $F_2$. So we have to process clauses between the length of $C_2$ and the length of $F_2$ - 1. Since we defined $C_2$ to be shorter than k and the maximum length of $F_2$ is $k$, we can derive $F_2$ without having to process a clause of length $k$ or greater.

        Since $F_1$ and $F_2$ are all possible cases of $F$ and they 
        can be derived without processing clauses longer than length
        $k-1$, $F$ can be derived by processing only clauses with
        a maximum length of $k-1$.
    \end{proof}

    \begin{lemma}
        Given the following:
        \begin{itemize}
            \item a clause $A$, of length less than k
            \item a clause $B$, of length less than k
            \item a clause $C$, of length k
            \item a clause $D$, of length k
            \item a clause $E$, of length k or k - 1
            \item $A$ expands to $C$ by lemma 5.8
            \item $B$ expands to $D$ by lemma 5.8
            \item $C$ and $D$ imply $E$ by lemma 5.7
        \end{itemize}
        Then $E$ can be derived by processing clauses whose length
        is at most k - 1.
    \end{lemma}
    \begin{proof}
        Consider the following figure:

        \begin{figure}[h]
            \includegraphics[scale=0.8]{519a}
            \caption{An illustration of the derivations described in this lemma}
            \Description[An illustration of the derivations described in this lemma]{An illustration of the derivations described in this lemma}            
        \end{figure}

        Let the clauses be defined as follows:

        $A := [a, b, \beta]$

        $B := [c, d, \delta]$

        Then the following clauses are derived:

        $C = [a, b, \beta, e, f, \phi]$

        $D = [c, d, \delta, g, h, \gamma]$

        $E = [a, b, \beta, e, f, \phi, c, d, \delta, g, h, \gamma]$

        Where $\beta, \delta, \phi,$ and $\gamma$ are fixed yet arbitrary
        set of terms in which the clauses block at least one assignment
        by Lemma 5.3.

        In order for C and D to imply E by Lemma 5.7, they have to be
        identical except for one term which is positive in one clause
        and negated in the other.

        There are 3 cases:
        (1) the opposite form term does not exist in A or B
        (2) the opposite form term exists in A or B
        (3) the opposite form term exists in A and B
        
        Consider case (1) the opposite form term does not exist in A or B.

        Then we fix one clause, say C, and redefine D in accordance with
        the requirements of Lemma 5.7. The clauses become:

        $C := [a, b, \beta, e, f, \phi]$

        $D := [a, b, \beta, -e, f, \phi]$

        $E = [a, b, \beta, f, \phi]$

        Notice that all of the terms in $A$ exist in $E$. Therefore, we
        can expand $A$, one term at a time by lemma 5.8, to derive
        $E$. Since a single term is added in each step, and $E$ has a maximum lengh of $k$,
        the largest clause that needs to be processed is of length k - 1.

        Consider case (2) the opposite form term exists in either A or B.

        Again since C and D are both generic clauses derived in the same
        way from other generic clauses, we can fix either. Let's fix C
        and redefine D. The clauses become:

        $C := [a, b, \beta, e, f, \phi]$

        $D := [-a, b, \beta, e, f, \phi]$

        $E = [b, \beta, e, f, \phi]$

        We know $-a$ does not exist in $B$ and $B$ expands to $D$.

        The terms in $B$ are therefore a subset of the terms in $D$ not counting
        $-a$.

        More clearly \{$c, d, \delta$\} $\subset$ \{$b, \beta, e, f, \phi$\}

        Notice the set on the right is just all of the terms in E.

        Therefore all of the terms in B exist in E and we can expand
        B to E by Lemma 5.8.

        The greatest possible length of B is k - 1 and we know only one
        term gets appended in each step by Lemma 5.8 so we only have to
        process clauses with a maximum length of k - 1 to derive E.

        Consider case (3) the opposite form term exists in A and B.

        In this case, we redefine the clauses as follows:

        $A := [a, b, \beta]$

        $B := [-a, d, \delta]$

        $C = [a, b, \beta, e, f, \phi]$

        $D = [-a, d, \delta, g, h, \gamma]$

        $E = [b, \beta, e, f, \phi, d, \delta, g, h, \gamma]$

        In this case, by Lemma 5.7, A and B can imply a new clause:

        $F = [b, \beta, d, \delta]$

        Want to show $F$ is shorter than $E$. 

        Consider two cases (3a) the length of $E$ is $k$ and (3b) the length of $E$ is $k-1$.

        Consider (3a) the length of $E$ is $k$.

        We can define $k$ as follows:

        $k = b + e + f + d + g + h 
            + \beta + \phi + \delta + \gamma
            - (\beta \phi) - (\beta \delta) - (\beta \gamma) - (\phi \delta) - (\phi \gamma) - (\delta \gamma)
            + (\beta \phi \delta) + (\beta \phi \gamma) + (\phi \delta \gamma)
            - (\beta \phi \delta \gamma)
            $

        And the length of $F$ is:

        Length of $F = b + d + \beta + \delta - (\beta \delta)$

        Want to show the length of $F$ is less than $k$:

        $b + d + \beta + \delta - (\beta \delta)$
        $<$
        $b + e + f + d + g + h 
            + \beta + \phi + \delta + \gamma
            - (\beta \phi) - (\beta \delta) - (\beta \gamma) - (\phi \delta) - (\phi \gamma) - (\delta \gamma)
            + (\beta \phi \delta) + (\beta \phi \gamma) + (\phi \delta \gamma)
            - (\beta \phi \delta \gamma)
        $

        $\rightarrow$
        $0$
        $<$
        $e + f + g + h 
            + \phi + \gamma
            - (\beta \phi) - (\beta \gamma) - (\phi \delta) - (\phi \gamma) - (\delta \gamma)
            + (\beta \phi \delta) + (\beta \phi \gamma) + (\phi \delta \gamma)
            - (\beta \phi \delta \gamma)
        $

        For ease, let's represent the generic sets of terms on the R.H.S. as $y$. The inequality becomes:

        $0 < e + f + g + h + y$

        Which is true as long as at least one term exists in {$e, f, g, h, y$}.

        Suppose not. Then no terms exist in {$e, f, g, h, y$}.

        We can redefine some clauses: 

        $C := [a, b, \beta]$

        Notice that no terms may exist ibn $\phi$ because any term in $\phi$ could be extracted and treated as $e$ or $f$, but we know these do not exist.

        Now we have the clauses:

        $C := [a, b, \beta]$

        $A := [a, b, \beta]$

        Notice the length of $C$ is the same as the length of $A$. Recall we defined the length of $C$ as $k$ and the length of $A$ as $k-1$. This is a contradiction so at least one term has to exist in {$e, f, g, h, y$}.

        Therefore the inequality holds and you can derive $E$ by processing clauses with a maximum length of $k-1$ if the length of $E$ is $k$.

        Consider (3b) the length of $E$ is $k - 1$.

        We can define $k$ as follows:

        $k = b + e + f + d + g + h 
            + \beta + \phi + \delta + \gamma
            - (\beta \phi) - (\beta \delta) - (\beta \gamma) - (\phi \delta) - (\phi \gamma) - (\delta \gamma)
            + (\beta \phi \delta) + (\beta \phi \gamma) + (\phi \delta \gamma)
            - (\beta \phi \delta \gamma)
            $

        And the length of $F$ is:

        Length of $F = b + d + \beta + \delta - (\beta \delta)$

        Want to show the length of $F$ is less than $k$:

        $b + d + \beta + \delta - (\beta \delta)$
        $<$
        $b + e + f + d + g + h 
            + \beta + \phi + \delta + \gamma
            - (\beta \phi) - (\beta \delta) - (\beta \gamma) - (\phi \delta) - (\phi \gamma) - (\delta \gamma)
            + (\beta \phi \delta) + (\beta \phi \gamma) + (\phi \delta \gamma)
            - (\beta \phi \delta \gamma)
        $

        Similarly to before, this will derive

        $0 < e + f + g + h + y + 1$

        Which is always true.

        Therefore the inequality holds and you can derive $E$ by processing clauses with a maximum length of $k-1$ if the length of $E$ is $k - 1$.

        Therefore both cases will allow you to derive $E$ by processing clauses of length no greater than $k$ - 1.        

        Since all three cases work, if you have clauses, A, B, C, D, and E, as described,
        then E can be derived by processing clauses with a maximum length
        of k - 1.
    \end{proof}

    \section{Algorithm}

    \begin{enumerate}
        \item For each clause in the instance, C, of length 3 or less:
        \begin{enumerate}
            \item For each clause in the instance, D, of length 3 or less:
            \begin{enumerate}
                \item Get all clauses implied by C and D according to Lemma 5.9 
                and add them to the instance
            \end{enumerate}
            \item For each clause in the instance, E, of length 1:
            \begin{enumerate}
                \item For each clause in the instance, F, of length 1:
                \begin{enumerate}
                    \item if E and F contain the same terminal in which it is 
                    positive in one clause and negated in the other, the 
                    clauses are contradicting and the instance is unsatisfiable, end
                \end{enumerate}
            \end{enumerate}
        \item Expand clauses to get all possible clauses of length 3 and add them to the instance
        \end{enumerate}
        \item Repeat (1) until no new clauses are added
        \item If it reaches here, the instance is satisfiable, end
    \end{enumerate}

    \section{Time Complexity Analysis}

    In this section I will analyze the time complexity of the algorithm in section 4

    (1) At most ${\binom{n}{3}}*8 + {\binom{n}{2}}*4 + {\binom{n}{1}}*2$ clauses to iterate which is on the order of

    $O(n^3)$

    (1.a) At most ${\binom{n}{3}}*8 + {\binom{n}{2}}*4 + {\binom{n}{1}}*2$ clauses to iterate which is on the order of

    $O(n^3)$

    (1.a.i) For each terminal in C, check if it's opposite form is in D

    $O(3^2)$ which is just constant time which is on the order of 

    $O(1)$

    (1.b) At most ${\binom{n}{1}} * 2$ clauses of length 1 exist, which is on the order of

    $O(n)$

    (1.b.i) At most ${\binom{n}{1}} * 2$ clauses of length 1 exist, which is on the order of

    $O(n)$

    (1.b.i.A) Constant time to check if two 1-terminal clauses contain the same terminal in the opposite form

    $O(1)$

    (1.c) For a 2-terminal clause, there are $2*n$ possible 3-terminal clauses that could be 
    expanded to. For a 1-terminal clause, there are $4*n^2$ possible 3-terminal clauses that
    could be expanded to. This is upper bounded by the latter case.

    $O(n^2)$

    (2) Worst case, we add one new clause each time so we have to do (1) ${\binom{n}{3}} * 8 + {\binom{n}{2}} * 8 + {\binom{n}{1}} * 2$ times which is on the order of

    $O(n^3)$

    (3) Constant time to check and return satisfiable

    $O(1)$

    The time complexity breaks down:

    $(2) * ((1) * ((1.a) * ((1.a.i)) + (1.b) * ((1.b.i) * (1.b.i.A)) + (1.c))) + (3)$

    $= O(n^3) * (O(n^3) * (O(n^3) * (O(1)) + O(n) * (O(n) * O(1)) + O(n^2))) + O(1)$

    $= O(n^3) * (O(n^3) * (O(n^3) * (O(1)) + O(n) * O(n) + O(n^2))) + O(1)$

    $= O(n^3) * (O(n^3) * (O(n^3) * (O(1)) + O(n^2) + O(n^2))) + O(1)$

    $= O(n^3) * (O(n^3) * (O(n^3) + O(n^2) + O(n^2))) + O(1)$

    $= O(n^3) * (O(n^3) * (O(n^3) + O(n^2))) + O(1)$

    $= O(n^3) * (O(n^3) * O(n^3)) + O(1)$

    $= O(n^3) * O(n^6) + O(1)$

    $= O(n^9) + O(1)$

    $= O(n^9)$

    \section{Proof of Correctness}

    Want to show an instance is unsatifiable iff we can derive contradicting
    1-terminal clauses by the algorithm.

    WTS Contradicting 1-terminal clauses can be derived $\implies$ the instance is unsatifiable

    Contradicting 1-terminal clauses take the form:

    $[a]$

    $[-a]$

    Where $a$ is a terminal in the problem.

    In all possible assignments, $a$ can have the value of True or False.

    If $a$ is True, the clause $[-a]$ will be False and the assignment does not satisfy the instance.
    If $a$ is False, the clause $[a]$ will be False and the assignment does not satisfy the instance.

    Since all possible assignments do not allow both clauses to be true, 
    the instance is unsatifiable.

    Want to show an unsatisfiable instance $\implies$ the algorithm will derive contradicting 1-terminal clauses

    By Lemma 5.14, the given 3-terminal clauses can be expanded to every
    possible $n$-terminal clause.

    By Lemma 5.15 these $n$-terminal clauses can be reduced by Lemma 5.7
    to derive contradicting 1-terminal clauses.

    So we know if we can derive these $n$-terminal clauses, we can derive
    contradicting 1-terminal clauses.

    The way in which we derive these 1-terminal clauses is we use
    Lemma 5.7 to reduce the current set of $k$-terminal clauses to
    a set of $(k-1)$-terminal clauses. We repeat for all $k$ from $n$ to $2$
    inclusive.

    Notice this passes through every possible $k$ from length $2$ to $n$.

    Recall that deriving all of the $n$-terminal clauses was done by using
    Lemma 5.8.

    These $n$-terminal clauses were then used to derive clauses of length $(n - 1)$
    by Lemma 5.7.

    By Lemma 5.18, such a case allows us to derive the clauses of length $(n - 1)$
    without ever having to process a clause of length $n$.

    Now we have all of the clauses of length $(n - 1)$ that we would have derived
    if we processed clauses of length $n$.

    Note how each of these clauses are either derived from given clauses
    by Lemma 5.8 or by Lemma 5.9 (all Lemma 5.7 derivations are a subset
    of all Lemma 5.9 derivations).

    We now want to use these $(n-1)$-terminal clauses to derive clauses
    of length $(n-2)$, but we want to do it without processing clauses
    whose length is larger than $(n-2)$.

    Since it takes two $(n-1)$-terminal clauses to derive a $(n-2)$-terminal
    clause by Lemma 5.7, the possible clauses could be of the form:

    \begin{enumerate}
        \item both clauses were derived by Lemma 5.8 (expansion)
        \item both clauses were derived by Lemma 5.9 (reduction/implication)
        \item each clause was derived in a different manner
    \end{enumerate}

    Note that this list is exhaustive because these are the only manner
    of implications used in the Lemmas that allowed us to derive these clauses.

    We can use the following lemmas to handle each case:

    \begin{enumerate}
        \item Lemma 5.19
        \item Lemma 5.17
        \item Lemma 5.18
    \end{enumerate}

    As such we can derive the clauses of length $(n - 2)$ without processing
    a clause whose length is greater than $(n - 2)$.

    In a more general sense, for every $w$ from $1$ to $n-4$, 
    we know we have clauses of length $(n-w)$ that we want to use to
    derive clauses of length $(n-w-1)$ and since we derived the clauses
    of length $(n-w)$ only using Lemma 5.9 (which is a more general form
    of 5.7) and Lemma 5.8, we know all clauses of length $(n-w)$ have to
    be derived one of those two ways. We use two $(n-w)$-terminal clauses to 
    derive a clause of length $(n-w-1)$ so the list of three ways these
    clauses could be derived is exhaustive. Since we have Lemmas 5.17, 
    5.18, and 5.19, all of these cases allow us to derive the clause
    of length $(n-w-1)$ without processing a clause longer than $(n-w-1)$.
    As such, we can derive all of the clauses we need down to clauses
    of length 4. Again all clauses of length 4 or greater were
    implied either by Lemma 5.8 (expansion) or Lemma 5.9 (reduction/implication).

    At this point, we have 4-terminal clauses and we want to derive 3-terminal
    clauses. We need two clauses and we can either include a 3-terminal clause
    or we cannot.

    If we don't include a 3-terminal clause, then the clauses used in the 
    implication were themselves derived and Lemmas 5.17-5.19 can be
    used to imply the 3-terminal clause without processing clauses
    of length 3 or higher.

    If we do use a 3-terminal clause, then the other clause has to be a
    4-terminal clause because we are using 4-terminal clause(s) to derive
    a 3-terminal clause,
    but Lemma's 5.17-5.19 only cover cases where the implying clauses were
    themselves derived. Since we are given 3-terminal clauses, there are some
    clauses which are not derived. Now we want to show any 3-terminal clause
    implied by a 4-terminal clause can be derived by processing clauses of length
    no more than 3. The 4-terminal clause can either be derived by Lemma 5.9 or
    expanded to by Lemma 5.8. In the former case, we can use Lemma 5.11 and in 
    the latter case we can use Lemma 5.12. In both cases, we can derive all the 
    necessary 3-terminal clauses while processing clauses with a maximum length
    of 3.

    Now we have all the 3-terminal clauses that would have been derived
    while the $n$-terminal clauses were being reduced to contradicting 1-terminal
    clauses.

    We can now reduce the 3-terminal clauses to 1-terminal clauses.

    Since we have all the necessary 3-terminal clauses without having
    to process a 4-terminal clause, this clearly shows we do not
    have to process any clauses of length 4 or greater to derive contradicting 1-terminal
    clauses.
    
    Even though the algorithm only explicitly uses Lemma 5.9 and Lemma 5.8, this will
    cover cases where reduction is needed because Lemma 5.9 is a more
    general case of Lemma 5.7. Notice, too, that Lemmas 5.11, 5.12, 5.17,
    5.18, and 5.19 rely on Lemmas 5.8 and 5.9 so we do not have to 
    explicitly capture the cases where the former lemmas would apply.

    Therefore, this coincides with the described algorithm.

    \section{Conclusion} 

    In this paper, we present an algorithm to solve 3SAT in polynomial time.

    This algorithm relies on Lemmas 5.8 and 5.9. The former of which says a clause
    can expand to imply another clause by adding on any term that's not in the original clause.
    The latter of which says two clauses sharing one terminal which is positive in one
    clause and negated in the other can imply a new clause composed of the rest of the terms
    in either clause. Using these strategies, we can process an instance of 3SAT 
    while only considering clauses of length 3 or less
    are guaranteed to derive a pair of contradicting 1-terminal clauses if and only if
    the instance is unsatisfiable.

    According to 
    \cite{Karp1972}, 3SAT is NP-complete, every problem in NP can be represented
    as an instance of 3SAT and such a conversion can happen in polynomial time.
    
    Since we have an algorithm to solve 3SAT in polynomial time, then all problems
    in NP can be solved in polynomial time.

    Thus, P = NP. 

    \bibliography{refs}
    \bibliographystyle{plain}

\end{document} 